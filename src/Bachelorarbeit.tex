\newcommand{\DeinName}{Pascal Fischer}
\newcommand{\MatrNo}{371778}
\newcommand{\TitelArbeit}{Entwicklung einer verschlüsselten Chat-basierten Plattform auf Basis des Matrix.org-Protokolls im Kontext von digitaler Gesundheit und Telemedizin}
\newcommand{\Fachgebiet}{Regelungssysteme}
\newcommand{\PrueferEins}{Dr.-Ing Thomas Schauer}
\newcommand{\PrueferZwei}{Prof.-Dr.-Ing. Clemens Gühmann}
\newcommand{\fachbereich}{Elektrotechnik und Informatik}

% entweder ein Datum h?ndisch eintragen, oder den Befehl \today nutzen
\newcommand{\Datum}{\today}

\makeglossary


%
% Einbinden des Headers, hier k?nnen auch weitere Einstellungen vorgenommen werden.
%%%%%%%%%%%%%%%%%%%%%%%%%%%%%%%%%%%%%%%%%%%%%%%%%%%%%%%
%																					%
%	In dieser Datei werden alle Packages eingebunden, 	%
% welche f�r das Dokument n�tig sind. Desweiteren 		%
% werden die Dokumentinformationen gesetzt.						%
%																											%
%%%%%%%%%%%%%%%%%%%%%%%%%%%%%%%%%%%%%%%%%%%%%%%%%%%%%%%
%
%	Die KOMAScript Dokumentklasse "scrbook" verwenden.
%
\documentclass[pdftex, 		%
							a4paper, 		% DIN A4 verwenden
							titlepage,	% separate Titelseite
							%draft,			%	Draft-Version, keine Bilder im pdf!
							final,			% Final-Version
							oneside,		% einseitiger Druck
							12pt,				% Schriftgr��e 12pt
							DIV=calc,
							%tocbasic,
							]{scrbook}	%	KOMAScript scrbook-Dokumentklasse

							\usepackage{geometry}
							\geometry{
								left=3cm,
								right=3cm,
								top=3cm,
								bottom=4cm,
								bindingoffset=5mm
							}

\usepackage{tikz}

%%%%%%%%%%%%%%%%%%%%%%%%%%%%%%%%%%%%%%%%%%%%%%%%%%%%%%%%
%	Einbinden der Pakete
%%%%%%%%%%%%%%%%%%%%%%%%%%%%%%%%%%%%%%%%%%%%%%%%%%%%%%%%

\usepackage[ngerman]{babel}

% PDF Dateien einbinden
\usepackage{pdfpages}

%Settings for PDF Pages to accept additonal versioned PDF files
\pdfminorversion=6
\pdfcompresslevel=9
\pdfobjcompresslevel=9

%Infos dazu unter: http://www.bakoma-tex.com/doc/latex/koma-script/scrhack.pdf
%Einige Pakete haben Probleme mit dem Komaskript.
\usepackage{scrhack}


% Definieren von eigenen benannten Farben.
% F�r sp�tere Verwendung in dem Dokument, definieren wir einzelne
% benannte Farben.
%
\usepackage{xcolor}
\definecolor{gray1}{gray}{0.92}
\definecolor{darkgreen}{rgb}{0,0.5,0}

\definecolor{urlLinkColor}{rgb}{0,0,0.5}
\definecolor{LinkColor}{rgb}{0,0,0}
\definecolor{ListingBackground}{rgb}{0.85,0.85,0.85}

\usepackage{pgf-umlsd}

\definecolor{delim}{RGB}{20,105,176}
\definecolor{numb}{RGB}{106, 109, 32}
\definecolor{string}{rgb}{0.64,0.08,0.08}
\rmfamily
\usepackage{amsfonts}
\usepackage[square,numbers]{natbib}
\usepackage{palatino}
\usepackage{amsmath}				% Schriftfamilie Palatino
%\usepackage[utf8]{inputenc}
%\usepackage[latin1]{inputenc} % Umlaute
\usepackage[utf8]{inputenc}
%\usepackage[dvips]{color}    	% f�r graue Boxen
%\usepackage[dvips]{graphicx} 	% Grafikpaket
\usepackage{multirow}
\usepackage{graphicx}
%\usepackage[table,xcdraw]{xcolor}
\usepackage{makeidx}   				% Paket zur Erzeugung eines Index
\usepackage{siunitx}
\sisetup{locale = DE}
\usepackage[normalem]{ulem}   % bietet Unterstreichungsvarianten
%\usepackage{picins} 					% Bilder im Absatz platzieren
\usepackage[T1]{fontenc}			% Erweiterten Zeichensatz aktivieren
\usepackage{multido}					% erm�glicht Schleifenartiges wiederholen von Befehlen
\usepackage{mdwlist}					% erm�glicht das Setzen des Z�hlers bei Aufz�hlungspunkten
\usepackage{paralist}					% Paket f�r Aufz�hlungen, erweitert Enumerate-Paket
\usepackage{longtable}				% mehrseitige Tabellen
\usepackage{tocbasic}
\parindent0pt           			% verzichte auf Einr�cken der ersten Zeile
\parskip1ex             			% Abstand zwischen den Abs�tzen

\usepackage{setspace}					% Paket zum Einstellen des Zeilenabstands
\onehalfspacing								% anderthalbfacher Zeilenabstand
%\doublespacing								% doppelter Zeilenabstand
%\singlespacing								% einfacher Zeilenabstand
\usepackage{float}
%\usepackage[german]{babel}
\usepackage[german=quotes]{csquotes} %Deutsche Anf�hrungszeichen
\usepackage{subfig}
\definecolor{LinkColor}{rgb}{0.1,0.1,0.1}
%\definecolor{ListingBackground}{rgb}{0.85,0.85,0.85}
\definecolor{ListingBackground}{rgb}{0.98,0.98,0.98}
\definecolor{gray}{rgb}{0.4,0.4,0.4}
\definecolor{darkblue}{rgb}{0.0,0.0,0.6}
\definecolor{cyan}{rgb}{0.0,0.6,0.6}



%
% Farbeinstellungen f�r die Links im PDF Dokument.
%
\makeindex

%-----------Paket f�r absolute Positionierung von Grafiken------------------
\usepackage[absolute]{textpos}
\setlength{\TPHorizModule}{1mm}
\setlength{\TPVertModule}{\TPHorizModule}

%-----------Aufz�hlungen und Einstellungen f�r Sourcecode-------------------
%\usepackage[savemem]{listings} %Bei wenig Arbeitsspeicher dies Option [savemem] aktivieren.
\usepackage{listings}
\lstloadlanguages{TeX,XML, Java} % TeX sprache laden, notwendig wegen option 'savemem'
\lstset{%
	language=[LaTeX]TeX,     % Sprache des Quellcodes ist TeX
	numbers=left,            % Zelennummern links
	stepnumber=1,            % Jede Zeile nummerieren.
	numbersep=5pt,           % 5pt Abstand zum Quellcode
	numberstyle=\tiny,       % Zeichengr�sse 'tiny' f�r die Nummern.
	breaklines=true,         % Zeilen umbrechen wenn notwendig.
	breakautoindent=true,    % Nach dem Zeilenumbruch Zeile einr�cken.
	postbreak=\space,        % Bei Leerzeichen umbrechen.
	tabsize=2,               % Tabulatorgr�sse 2
	basicstyle=\ttfamily\footnotesize, % Nichtproportionale Schrift, klein f�r den Quellcode
	showspaces=false,        % Leerzeichen nicht anzeigen.
	showstringspaces=false,  % Leerzeichen auch in Strings ('') nicht anzeigen.
	extendedchars=true,      % Alle Zeichen vom Latin1 Zeichensatz anzeigen.
	backgroundcolor=\color{ListingBackground}} % Hintergrundfarbe des Quellcodes setzen.


\lstset{
  basicstyle=\small\ttfamily,
  columns=fullflexible,
  showstringspaces=false,
  %commentstyle=\color{gray}\upshape
}
%neue Lang definieren, als Bsp.
\lstdefinelanguage{XML-changed}
{
  basicstyle=\footnotesize\ttfamily\bfseries,
  morestring=[b]",
  morestring=[s]{>}{<},
  morecomment=[s]{<?}{?>},
  stringstyle=\color{black},
  identifierstyle=\color{darkblue},
  keywordstyle=\color{cyan},
  morekeywords={xmlns,version,type}% list your attributes here
}

%-----------Caption Package-------------------
\usepackage{caption}
\DeclareCaptionFont{white}{\color{white}}
\DeclareCaptionFormat{listing}{\colorbox[cmyk]{0.43, 0.35, 0.35,0.01}{\parbox{\textwidth}{\hspace{15pt}#1#2#3}}}

\DeclareCaptionFormat{graphics}{\colorbox[cmyk]{0.43, 0.35, 0.35,0.01}{\parbox{\textwidth}{\hspace{15pt}#1#2#3}}}


\captionsetup[lstlisting]{format=listing,labelfont=white,textfont=white, singlelinecheck=false, margin=0pt, font={footnotesize}}

%-----------Header+Footer---------------------------------------------------
\usepackage{fancyhdr}					%
\pagestyle{fancy}							%

\fancyhead{}
\fancyfoot{}
\renewcommand{\headrulewidth}{0.4pt} % Kopzeilenlinie
\renewcommand{\footrulewidth}{0.0pt} % Fusszeilenlinie 0.0pt blendet sie aus

\renewcommand{\chaptermark}[1]{\markboth{\thechapter\quad#1}{}}
\renewcommand{\sectionmark}[1]{\markright{\thesection\quad#1}}

%\fancyhead[LO]{\small\sffamily\rightmark}
\fancyhead[LE,RO]{\slshape \rightmark}
\fancyhead[LO,RE]{\slshape \leftmark}
%\fancyhead[RO]{\small\sffamily\thepage}
\fancyfoot[C]{\thepage}

%-----------Um die Eidesstattliche Erkl�rung als PDF einzubinden-----------------------------------
%\usepackage{pdfpages}

%------------Glossar--------------------------------------------------------------
\usepackage{expdlist}
\usepackage{/home/pfischer/Uni/MyPaper/src/glossar}

\renewcommand{\glshead}{\chapter*{Glossar}}
\renewcommand{\glentry}[2]{\glossary{#1@[#1] #2|glspage}}
\renewcommand{\glsgroup}[1]{{\listpart{\makebox[0pt][l]{\rule[-2pt]{\textwidth}{0.5pt}}{\textbf{\large #1}}}}}

\makeglossary

%Glossar mit Bordmitteln ------------------------------------------------------------------
%Darstellung des Glossars einstellen
%\usepackage[style=super, header=none, border=none, number=none, cols=2, toc=true]{glossary}
%\renewcommand{\glossaryname}{Glossar}
%\printglossary

% --- diverse Schriften -------------------------------------------------------------------
%\newcommand{\url}[1]{{\sf\small #1}}     % Hyperlinks


% -------F�r ToDo-Notes--------------------------------------------------------------------
\usepackage[color=red, shadow]{todonotes} % ", disable" deaktiviert ToDo-Notes
%Vereinfachtes "Inline-Todo"
\newcommand{\td}[1]{{\todo[inline]{#1}}}
\newcommand{\tdu}[1]{{\todo[inline, color=green!40]{#1}}}

\newcommand{\SubItem}[1]{
	{\setlength\itemindent{15pt} \item[-] #1}
}
\newcommand{\citepnum}[1]{[\citenum{#1}]}

%--------F�r Links-------------------------------------------------------------------------




%--------HyperRef konfigurieren-------------------------------------------------------------------------

\usepackage[
	pdftitle={\TitelArbeit},
	pdfauthor={\DeinName},
	pdfsubject={\TitelArbeit},
	pdfcreator={MiKTeX, LaTeX with hyperref and KOMA-Script auf Basis der Vorlage von seiler.it},
	pdfkeywords={Abschlussarbeit, TU Berlin, Universität},%weitere Keywords hier einf�gen
	pdfpagemode=UseOutlines,%
	pdfdisplaydoctitle=true,%
	pdflang=de%
]{hyperref}

\hypersetup{%
	colorlinks=true,%        Aktivieren von farbigen Links im Dokument (keine Rahmen)
	linkcolor=LinkColor,%    Farbe festlegen.
	citecolor=LinkColor,%    Farbe festlegen.
	filecolor=LinkColor,%    Farbe festlegen.
	menucolor=LinkColor,%    Farbe festlegen.
	urlcolor=LinkColor,%     Farbe von URL's im Dokument.
	bookmarksnumbered=true%  �berschriftsnummerierung im PDF Inhalt anzeigen.
}

\newcommand{\improvement}{\todo[inline, color=green]}
\newcommand{\change}{\todo[inline, color=blue]}
\newcommand{\unsure}{\todo[inline, color=violet]}
\newcommand{\eunorm}[1]{\lVert #1 \rVert_{\num{2}}}
\newcommand{\anfzch}[1]{\glqq #1\grqq{}}


\usepackage{enumitem}

\usepackage{tabularray}

\usepackage{wrapfig}

\usepackage{acronym}

\usepackage{hyphenat}
% --------- Generates Chapter header ------------------------

\DeclareMathAlphabet\EuRoman{U}{eur}{m}{n}
\SetMathAlphabet\EuRoman{bold}{U}{eur}{b}{n}
\newcommand{\eurom}{\EuRoman}
\definecolor{nicered}{rgb}{.647,.129,.149}
\usepackage{etoolbox}

\makeatletter
\newsavebox{\feline@chapter}
\newcommand\feline@chapter@marker[1][4cm]{%
	\sbox\feline@chapter{%
		\resizebox{!}{#1}{\setlength{\fboxsep}{0pt}%
		\colorbox{white}{\color{black}$\eurom\thechapter$}}}%
	\raisebox{\depth}{\usebox{\feline@chapter}}%
}
\renewcommand*{\chapterformat}{%
	\sbox\feline@chapter{\feline@chapter@marker[1.61cm]}%
	\makebox[0pt][l]{%
		\makebox[\dimexpr\textwidth][r]{%
			\usebox\feline@chapter}}%
}
\makeatother

\preto\chapterheadendvskip{%
	\vspace*{-\parskip}%
	{\noindent\setlength\parfillskip{0pt plus 1fil}\rule{\textwidth}{.4pt}\par}%
}


\lstdefinelanguage{json}{
    numbers=left,
    numberstyle=\small,
    frame=single,
    rulecolor=\color{black},
    showspaces=false,
    showtabs=false,
    breaklines=true,
    postbreak=\raisebox{0ex}[0ex][0ex]{\ensuremath{\color{gray}\hookrightarrow\space}},
    breakatwhitespace=true,
    basicstyle=\ttfamily\small,
    upquote=true,
    morestring=[b]",
    stringstyle=\color{string},
    literate=
    *{0}{{{\color{numb}0}}}{1}
        {1}{{{\color{numb}1}}}{1}
        {2}{{{\color{numb}2}}}{1}
        {3}{{{\color{numb}3}}}{1}
        {4}{{{\color{numb}4}}}{1}
        {5}{{{\color{numb}5}}}{1}
        {6}{{{\color{numb}6}}}{1}
        {7}{{{\color{numb}7}}}{1}
        {8}{{{\color{numb}8}}}{1}
        {9}{{{\color{numb}9}}}{1}
        {\{}{{{\color{delim}{\{}}}}{1}
        {\}}{{{\color{delim}{\}}}}}{1}
        {[}{{{\color{delim}{[}}}}{1}
        {]}{{{\color{delim}{]}}}}{1},
}

% Beginn des Dokuments
\begin{document}

%Zitiert alle Referenzen, ohne Sie hier zu listen. Dadurch erscheinen alle Quellen im %Literaturverzeichnis, auch wenn sie im Text nicht genutzt werden.
%Bitte nur f?r Testzwecke verwenden.
    \nocite{*}


% Einbinden des Deckblatts

    \frontmatter
    % erste Seite (Titelseite) der Diplomarbeit mit Titel, Name usw...

\newsavebox{\Prof}
\savebox{\Prof}{Erster Prüfer }

\newsavebox{\Betr}
\savebox{\Betr}{Zweiter Prüfer }
\begin{titlepage}

%\begin{minipage}[h]{0.5\textwidth}
%		\raggedright
%		\includegraphics[height=4cm]{source/images/Logo/TU-Berlin-Logo.png}
%	\end{minipage}
%	\begin{minipage}[h]{0.5\textwidth}
%		\raggedleft
%		\includegraphics[height=5cm]{source/images/Logo/sensorstim_logo.png}
%	\end{minipage}


	\begin{figure}
		\begin{center}
			\includegraphics[height=4cm]{source/images/Logo/TU_Berlin_Logo.png}
		\end{center}
	\end{figure}

	\begin{center}
%\includegraphics[height=2cm]{source/images/logo_mirolab.png}

	{\Large Technische Universität Berlin}\\[1mm]
	{\large Fakultät Elektrotechnik und Informatik}\\
	{\large Fachgebiet \Fachgebiet}\\
	\vspace{1cm}
	{\LARGE  Bachelorarbeit}

	\vspace{0.5cm}

	{\LARGE\textbf{ \TitelArbeit}}\\


	\vspace{1.cm}

	von\\[2mm]

	\textbf{\large{\DeinName}}\\
%\vspace{1.5cm}
	Matrikelnummer: \MatrNo\\[1cm]

	Betreuer:
	Dr. Thomas Schauer,
	Ardit Dvorani\\[2mm]

	\usebox{\Prof: \PrueferEins} \\
	\usebox{\Betr: \PrueferZwei} \\

	\vspace{1cm}
	Berlin, \today
	\end{center}
\end{titlepage}
    \hypersetup{pageanchor=false}

    \newpage
    \section*{Zusammenfassung}
    In dieser Arbeit befasse ich mich mit ....

    \section*{Abstract}\label{sec:abstract}
    This paper describes how to ...
    \newpage
    \chapter*{Eidesstattliche Erklärung}\label{ch:eidesstattliche-erklarung}
    Hiermit erkläre ich, dass ich die vorliegende Arbeit selbstständig und eigenhändig sowie ohne
    unerlaubte fremde Hilfe und ausschließlich unter Verwendung der aufgeführten Quellen und
    Hilfsmittel angefertigt habe.\\
    \vspace{1cm}\\
    Berlin, den \underline{\hspace{3cm}} \hfill \DeinName \underline{\hspace{4cm}}

    \newpage
    \tableofcontents

    \mainmatter

    \newpage
    \chapter{Einleitung}\label{ch:einleitung}

    \section{Motivation}\label{sec:motivation}
    Mit voranschreiten der Digitalisierung
    Im Zuge der Coronapandemie im Jahre 2019 wurden nicht nur die Rufe nach

    \section{Zielsetzung}\label{sec:zielsetzung}
    Ziel dieser Arbeit ist es eine Plattform zu entwickeln welchen es Ärzten und Patienten ermöglicht in Kontakt zu treten und Nachrichten auszutauschen.
    Hierfür soll eine Chat-App auf basis des Matrix.org-Protokolls implementiert werden.
    Die App ist in der Programmiersprache Swift zu entwickeln und soll auf allen gängigen iOS geräten unterstützt werden.
    Darüber hinaus muss die benötigte Infrastruktur bereitgestellt werden um Nachrichten und Daten zu speichern und zu übermitteln.
    Da medizinische Daten dem Datenschutz in besonderer Maße unterliegen ist es besonders wichtig,
    dass jegliche Art von Kommunikation zwischen Ärzten und Patienten stets verschlüsselt ist.
    Es muss gewährleistet sein, dass zu jedem Zeitpunkt alle personenbezogenen Daten vor dem Zugriff unerlaubter Dritter geschützt sind.
    Da es im Bereich der Medizin  hinaus ist zu beachten, dass das Userinterface
    So sollte beispielsweise darauf geachtet werden, dass Da es im Bereich der Medizin hinaus ist zu beachten, dass das Userinterface
    So sollte beispielsweise darauf geachtet werden, dass

    \section{Aufbau der Arbeit}\label{sec:aufbau-der-arbeit}


    \newpage
    \chapter{Theoretische Grundlagen}\label{ch:theoretische-grundlagen}
    In diesem Kapitel werden zunächst die theoretischen Grundlagen, welche die Implementierung der Plattform beeinflusst haben, vorgestellt.
    Anschließend wird das Matrix Protokoll genauer betrachtet und das Grundprinzip erklärt.
    Desweiteren werden die in der End-to-End Verschlüsselung genutzten keys erläutert und genauer auf die verschiedenen Verschlüsselungsalgorithmen eingegangen.

    \section{Datenschutz im Gesundheitswesen}
    Die Rechtlichen Rahmenbedingungen für das Speicher und Verarbeiten medizinischer Daten sind in der EU-Datenschutz-Grundverordnung\footnote{https://dsgvo-gesetz.de/} (DSGVO) festgelegt.
    Diese fordert ebenfalls zur Einhaltung des IT-Sicherheitsgesetzes und des E-Health-Gesetzes auf.
    Medizinische Daten werden vom Gesetzgeber als besondere Arten personenbezogener Daten eingeordnet, da sie besonders sensible Informationen enthalten.~\cite{datenschutzimgesundeitswesen}

    \textit{Das am 29. Dezember 2015 in Kraft getretene "Gesetz für sichere digitale Kommunikation und Anwendungen im Gesundheitswesen (E-Health-Gesetz)" hat die ersten Weichen für den Aufbau der sicheren Telematikinfrastruktur (TI) und die Einführung medizinischer digitaler Anwendungen gestellt.
    Ziel dieses Gesetzes war es, die Chancen der Digitalisierung für die Gesundheitsversorgung zu nutzen und eine schnelle Einführung medizinischer Anwendungen für die Patientinnen und Patienten zu ermöglichen.}~\cite{ehealthgesetz}

    Ein Teil der Telematikinfrastruktur ist der TI-Messenger der Gematik GmbH.

    \newpage
    \section{Matrix}\label{sec:matrix}

    \subsection{Matrix Protokoll}\label{subsec:matrix-protokoll}
    Das Matrix Protokoll ist ein offener Standard für dezentrale Echtzeitkommunikation im Internet.
    Es ist ein Open-Source-Projekt welches unter Aufsicht der Matrix.org Foundation, einer Non-Profit-Organisation aus England, entwickelt wird.
    Das Team besteht aus 12 Personen, viele von ihnen mit weitreichenden Erfahrungen im Bereich VoIP- und Messaging-Apps für Mobilgeräte.
    Die Entwicklung begann im Jahr 2014 und ist seit Juni 2019 für den Einsatz im Produktionsbetrieb geeignet.
    Mittlerweile gibt es zahlreiche Beiträge aus der Community, wodurch das Projekt weiter vorangetrieben wird.
    ~\cite{matrixfaq}

    Das ursprüngliche Ziel des Projektes war es ein Protokoll zu entwickeln, welches es Usern ermöglichen soll, anderen Usern Nachrichten zu schreiben oder diese anzurufen, unabhängig davon welche Platform diese benutzen.
    Langfristig soll Matrix ein generisches System zur Messaging und Datensynchronisation über HTTP für das ganze Internet bilden.
    ~\cite{matrixfaq}

    Hierzu definiert Matrix eine Reihe von REST-APIs:
    \begin{description}[leftmargin=!,labelwidth=5.5cm]
        \item [Client-Server-API\footnotemark] \footnotetext{https://spec.matrix.org/v1.4/client-server-api/} zur Kommunikation zwischen Matrix kompatiblen Clients und einem Homeserver
        \item [Server-Server-API\footnotemark] \footnotetext{https://spec.matrix.org/v1.4/server-server-api/} zum Nachrichtenaustausch und Synchronisation zwischen mehreren Homservern
        \item [Application-Service-API\footnotemark] \footnotetext{https://spec.matrix.org/v1.4/application-service-api/} zur Erweiterung der Funktionalität von Matrix und Integration anderer Systeme
        \item [Identity-Service-API\footnotemark] \footnotetext{https://spec.matrix.org/v1.4/identity-service-api/} beschreibt das Mapping zischen Drittanbietern
        \item [Push-Gateway-API\footnotemark] \footnotetext{https://spec.matrix.org/v1.4/push-gateway-api/} um versenden von Push Notifications an Clients falls neue Events beim Homeserver eintreffen
    \end{description}



    \subsection{Funktionsweise}\label{subsec:funktionsweise}
%
%\usetikz
%\begin{figure}
%    \centering
%    

% Pattern Info

\tikzset{
    pattern size/.store in=\mcSize,
    pattern size = 5pt,
    pattern thickness/.store in=\mcThickness,
    pattern thickness = 0.3pt,
    pattern radius/.store in=\mcRadius,
    pattern radius = 1pt}
\makeatletter
\pgfutil@ifundefined{pgf@pattern@name@_2eflf89xd}{
    \pgfdeclarepatternformonly[\mcThickness,\mcSize]{_2eflf89xd}
        {\pgfqpoint{0pt}{0pt}}
        {\pgfpoint{\mcSize}{\mcSize}}
        {\pgfpoint{\mcSize}{\mcSize}}
        {
        \pgfsetcolor{\tikz@pattern@color}
        \pgfsetlinewidth{\mcThickness}
        \pgfpathmoveto{\pgfqpoint{0pt}{\mcSize}}
        \pgfpathlineto{\pgfpoint{\mcSize+\mcThickness}{-\mcThickness}}
        \pgfpathmoveto{\pgfqpoint{0pt}{0pt}}
        \pgfpathlineto{\pgfpoint{\mcSize+\mcThickness}{\mcSize+\mcThickness}}
        \pgfusepath{stroke}
    }}
\makeatother

% Pattern Info

\tikzset{
    pattern size/.store in=\mcSize,
    pattern size = 5pt,
    pattern thickness/.store in=\mcThickness,
    pattern thickness = 0.3pt,
    pattern radius/.store in=\mcRadius,
    pattern radius = 1pt}
\makeatletter
\pgfutil@ifundefined{pgf@pattern@name@_a6o28e1ro}{
    \pgfdeclarepatternformonly[\mcThickness,\mcSize]{_a6o28e1ro}
        {\pgfqpoint{0pt}{0pt}}
        {\pgfpoint{\mcSize+\mcThickness}{\mcSize+\mcThickness}}
        {\pgfpoint{\mcSize}{\mcSize}}
        {
        \pgfsetcolor{\tikz@pattern@color}
        \pgfsetlinewidth{\mcThickness}
        \pgfpathmoveto{\pgfqpoint{0pt}{0pt}}
        \pgfpathlineto{\pgfpoint{\mcSize+\mcThickness}{\mcSize+\mcThickness}}
        \pgfusepath{stroke}
    }}
\makeatother

% Pattern Info

\tikzset{
    pattern size/.store in=\mcSize,
    pattern size = 5pt,
    pattern thickness/.store in=\mcThickness,
    pattern thickness = 0.3pt,
    pattern radius/.store in=\mcRadius,
    pattern radius = 1pt}
\makeatletter
\pgfutil@ifundefined{pgf@pattern@name@_zxxkiapd2}{
    \makeatletter
    \pgfdeclarepatternformonly[\mcRadius,\mcThickness,\mcSize]{_zxxkiapd2}
        {\pgfpoint{-0.5*\mcSize}{-0.5*\mcSize}}
        {\pgfpoint{0.5*\mcSize}{0.5*\mcSize}}
        {\pgfpoint{\mcSize}{\mcSize}}
        {
        \pgfsetcolor{\tikz@pattern@color}
        \pgfsetlinewidth{\mcThickness}
        \pgfpathcircle\pgfpointorigin{\mcRadius}
        \pgfusepath{stroke}
    }}
\makeatother

% Pattern Info

\tikzset{
    pattern size/.store in=\mcSize,
    pattern size = 5pt,
    pattern thickness/.store in=\mcThickness,
    pattern thickness = 0.3pt,
    pattern radius/.store in=\mcRadius,
    pattern radius = 1pt}
\makeatletter
\pgfutil@ifundefined{pgf@pattern@name@_7ksk1ug9w}{
    \pgfdeclarepatternformonly[\mcThickness,\mcSize]{_7ksk1ug9w}
        {\pgfqpoint{0pt}{0pt}}
        {\pgfpoint{\mcSize}{\mcSize}}
        {\pgfpoint{\mcSize}{\mcSize}}
        {
        \pgfsetcolor{\tikz@pattern@color}
        \pgfsetlinewidth{\mcThickness}
        \pgfpathmoveto{\pgfqpoint{0pt}{\mcSize}}
        \pgfpathlineto{\pgfpoint{\mcSize+\mcThickness}{-\mcThickness}}
        \pgfpathmoveto{\pgfqpoint{0pt}{0pt}}
        \pgfpathlineto{\pgfpoint{\mcSize+\mcThickness}{\mcSize+\mcThickness}}
        \pgfusepath{stroke}
    }}
\makeatother

% Pattern Info

\tikzset{
    pattern size/.store in=\mcSize,
    pattern size = 5pt,
    pattern thickness/.store in=\mcThickness,
    pattern thickness = 0.3pt,
    pattern radius/.store in=\mcRadius,
    pattern radius = 1pt}
\makeatletter
\pgfutil@ifundefined{pgf@pattern@name@_qh7wu9oqq}{
    \pgfdeclarepatternformonly[\mcThickness,\mcSize]{_qh7wu9oqq}
        {\pgfqpoint{0pt}{0pt}}
        {\pgfpoint{\mcSize+\mcThickness}{\mcSize+\mcThickness}}
        {\pgfpoint{\mcSize}{\mcSize}}
        {
        \pgfsetcolor{\tikz@pattern@color}
        \pgfsetlinewidth{\mcThickness}
        \pgfpathmoveto{\pgfqpoint{0pt}{0pt}}
        \pgfpathlineto{\pgfpoint{\mcSize+\mcThickness}{\mcSize+\mcThickness}}
        \pgfusepath{stroke}
    }}
\makeatother

% Pattern Info

\tikzset{
    pattern size/.store in=\mcSize,
    pattern size = 5pt,
    pattern thickness/.store in=\mcThickness,
    pattern thickness = 0.3pt,
    pattern radius/.store in=\mcRadius,
    pattern radius = 1pt}
\makeatletter
\pgfutil@ifundefined{pgf@pattern@name@_2e10vbjsb}{
    \makeatletter
    \pgfdeclarepatternformonly[\mcRadius,\mcThickness,\mcSize]{_2e10vbjsb}
        {\pgfpoint{-0.5*\mcSize}{-0.5*\mcSize}}
        {\pgfpoint{0.5*\mcSize}{0.5*\mcSize}}
        {\pgfpoint{\mcSize}{\mcSize}}
        {
        \pgfsetcolor{\tikz@pattern@color}
        \pgfsetlinewidth{\mcThickness}
        \pgfpathcircle\pgfpointorigin{\mcRadius}
        \pgfusepath{stroke}
    }}
\makeatother
\tikzset{every picture/.style={line width=0.75pt}} %set default line width to 0.75pt

\begin{tikzpicture}[x=0.75pt,y=0.75pt,yscale=-1,xscale=1]
%uncomment if require: \path (0,580); %set diagram left start at 0, and has height of 580

%Shape: Ellipse [id:dp9373687458561233]
    \draw  [pattern=_2eflf89xd,pattern size=6pt,pattern thickness=0.75pt,pattern radius=0pt, pattern color={rgb, 255:red, 0; green, 0; blue, 0}] (246.86,261.57) .. controls (246.86,241.87) and (262.73,225.89) .. (282.31,225.89) .. controls (301.89,225.89) and (317.76,241.87) .. (317.76,261.57) .. controls (317.76,281.28) and (301.89,297.25) .. (282.31,297.25) .. controls (262.73,297.25) and (246.86,281.28) .. (246.86,261.57) -- cycle ;
%Shape: Ellipse [id:dp21917683521442988]
    \draw  [pattern=_a6o28e1ro,pattern size=6pt,pattern thickness=0.75pt,pattern radius=0pt, pattern color={rgb, 255:red, 0; green, 0; blue, 0}] (362.46,194.09) .. controls (362.46,174.39) and (378.33,158.41) .. (397.91,158.41) .. controls (417.49,158.41) and (433.36,174.39) .. (433.36,194.09) .. controls (433.36,213.8) and (417.49,229.77) .. (397.91,229.77) .. controls (378.33,229.77) and (362.46,213.8) .. (362.46,194.09) -- cycle ;
%Shape: Ellipse [id:dp35337222919510536]
    \draw  [pattern=_zxxkiapd2,pattern size=6pt,pattern thickness=0.75pt,pattern radius=0.75pt, pattern color={rgb, 255:red, 0; green, 0; blue, 0}] (371.71,335.25) .. controls (371.71,315.55) and (387.58,299.58) .. (407.16,299.58) .. controls (426.74,299.58) and (442.61,315.55) .. (442.61,335.25) .. controls (442.61,354.96) and (426.74,370.93) .. (407.16,370.93) .. controls (387.58,370.93) and (371.71,354.96) .. (371.71,335.25) -- cycle ;
%Shape: Ellipse [id:dp012489074208077877]
    \draw  [pattern=_7ksk1ug9w,pattern size=6pt,pattern thickness=0.75pt,pattern radius=0pt, pattern color={rgb, 255:red, 0; green, 0; blue, 0}] (159,260.41) .. controls (159,251.2) and (166.42,243.73) .. (175.57,243.73) .. controls (184.72,243.73) and (192.14,251.2) .. (192.14,260.41) .. controls (192.14,269.62) and (184.72,277.08) .. (175.57,277.08) .. controls (166.42,277.08) and (159,269.62) .. (159,260.41) -- cycle ;
%Shape: Ellipse [id:dp629970923438828]
    \draw  [pattern=_qh7wu9oqq,pattern size=6pt,pattern thickness=0.75pt,pattern radius=0pt, pattern color={rgb, 255:red, 0; green, 0; blue, 0}] (448.01,124.68) .. controls (448.01,115.47) and (455.43,108) .. (464.58,108) .. controls (473.73,108) and (481.15,115.47) .. (481.15,124.68) .. controls (481.15,133.89) and (473.73,141.35) .. (464.58,141.35) .. controls (455.43,141.35) and (448.01,133.89) .. (448.01,124.68) -- cycle ;
%Shape: Ellipse [id:dp16375813237618053]
    \draw  [pattern=_2e10vbjsb,pattern size=6pt,pattern thickness=0.75pt,pattern radius=0.75pt, pattern color={rgb, 255:red, 0; green, 0; blue, 0}] (451.86,409.32) .. controls (451.86,400.11) and (459.28,392.65) .. (468.43,392.65) .. controls (477.58,392.65) and (485,400.11) .. (485,409.32) .. controls (485,418.53) and (477.58,426) .. (468.43,426) .. controls (459.28,426) and (451.86,418.53) .. (451.86,409.32) -- cycle ;
%Straight Lines [id:da21697884743721452]
    \draw    (192.14,260.41) -- (246.86,261.57) ;
%Straight Lines [id:da43988807724258416]
    \draw    (313.91,243.73) -- (365.54,212.71) ;
%Straight Lines [id:da3575969507846568]
    \draw    (374.02,318.19) -- (312.37,280.96) ;
%Straight Lines [id:da3537393507838251]
    \draw    (401.77,230.55) -- (407.16,299.58) ;
%Straight Lines [id:da558329025283854]
    \draw    (451.86,136.7) -- (423.35,166.95) ;
%Straight Lines [id:da3588367968445647]
    \draw    (428.74,363.18) -- (458.03,396.53) ;

% Text Node
    \draw (33,250) node [anchor=north west][inner sep=0.75pt]   [align=left] {@alice:alice.com};
% Text Node
    \draw (490,114) node [anchor=north west][inner sep=0.75pt]   [align=left] {@bob:bob.com};
% Text Node
    \draw (495,401) node [anchor=north west][inner sep=0.75pt]   [align=left] {@charlie:charlie.com};
% Text Node
    \draw (424,282) node [anchor=north west][inner sep=0.75pt]   [align=left] {matrix.charlie.com};
% Text Node
    \draw (326,131) node [anchor=north west][inner sep=0.75pt]   [align=left] {matrix.bob.com};
% Text Node
    \draw (226,198) node [anchor=north west][inner sep=0.75pt]   [align=left] {matrix.alice.com};


\end{tikzpicture}

%    \caption{Beispielsituation}
%    \label{fig:matrixfunktionsweise}
%\end{figure}

    \subsection{Events}\label{sec:events}
    Im Matrix Protokoll werden alle Ereignisse (Events) in einer einzigen großen Kette (Timeline) gespeichert. Diese Events können in zwei Kategorien unterteilt werden. Zum einen gibt es State Events, welche Ereginsse wie die Erzeugung eines Raumes oder das Beitreten eines Users zu einem Raum beschreiben.
    Zum anderen gibt es Message Events welche realen Datenaustausch zwischen Usern beschreiben. Dies können beispielsweise Textnachrichten, Fotos oder Videos sein. Jedes Event enthält unter anderem folgende Attribute:
    \begin{description}[leftmargin=!,labelwidth=3.5cm]
        \item [event\_id] ein eindeutiger Identifier für das jeweilige Event
        \item [sender] die Matrix ID des Absenders des Events
        \item [room\_id] ein eindeutiger Identifier über welchen ein Event einem bestimmten Raum zugeordnet werden kann
        \item [type] der Typ des Events
        \item [content] der tatsächliche Inhalt der Nachricht. Dieser variiert abhängig vom event Type
        \item [origin\_server\_ts] Zeitstempel
    \end{description}

    \subsection{Rooms}


    \begin{lstlisting}[language=json,firstnumber=1]
{"menu": {
  "id": "file",
  "value": "File",
  "popup": {
    "menuitem": [
      {"value": "New", "onclick": "CreateNewDoc()"},
      {"value": "Open", "onclick": "OpenDoc()"},
      {"value": "Close", "onclick": "CloseDoc()"}
    ]
  }
}}
    \end{lstlisting}


    \subsection{Devices}


    \subsection{Verwendete Schlüssel}\label{subsec:verwendete-schlussel}
    text


    \section{Verschlüsselung}\label{sec:verschlusselung}

    \subsection{Ed25519}\label{subsec:ed25519}
    text

    \subsection{Curve25519}\label{subsec:curve25519}
    text

    \subsection{AES-256}\label{subsec:aes-256}
    text

    \subsection{HMAC-SHA-256}\label{subsec:hmac-sha-256}
    text

    \section{REST-API}\label{sec:rest}

    \cite{dazer2012restful}

    \newpage
    \section{Anwendungsarchitektur}\label{sec:anwendungsarchitektur}
    Wie im Bereich von Webapplications haben sich auch im Bereich von Mobile Apps über die Zeit bestimmte Anwendungsarchitekturen (Design Patterns) entwickelt.
    Diese dienen als Muster um wiederkehrende Entwurfsprobleme zu lösen, indem man auf bewährte Praktiken zurückgreift.
    Sie beschreiben die allgemeine Struktur der Anwendung und geben somit auch die Aufteilung des Quellcodes vor.
    Die richtige Wahl der Anwendungsarchitektur kann einen entscheidenden Einfluss darauf nehmen wie effizient die Entwicklung des Projektes verläuft.
    Darüber hinaus erleichtern sie die Arbeit im Team da sie ein einheitliches Vokabular bieten.
    Im Folgenden werden 3 der bekanntesten Architekturen für Mobile Apps näher erläutert.

    \newpage
    \subsection{Model-View-Controller}\label{subsec:model-view-controller}
    Die Model-View-Controller Architektur teilt eine Anwendung in 3 Bereiche auf welche in Abbildung~\ref{fig:mvc} dargestellt sind.

    \begin{figure}[h]
        \centering
        \tikzset{every picture/.style={line width=0.75pt}} %set default line width to 0.75pt

\begin{tikzpicture}[x=0.75pt,y=0.75pt,yscale=-1,xscale=1]
    %uncomment if require: \path (0,300); %set diagram left start at 0, and has height of 300

    %Rounded Rect [id:dp23143765937354965]
    \draw   (301.1,18.32) .. controls (301.1,12.08) and (306.16,7.02) .. (312.4,7.02) -- (390.8,7.02) .. controls (397.04,7.02) and (402.1,12.08) .. (402.1,18.32) -- (402.1,52.22) .. controls (402.1,58.46) and (397.04,63.52) .. (390.8,63.52) -- (312.4,63.52) .. controls (306.16,63.52) and (301.1,58.46) .. (301.1,52.22) -- cycle ;
    %Rounded Rect [id:dp11583586667477586]
    \draw   (480,211.3) .. controls (480,205.06) and (485.06,200) .. (491.3,200) -- (569.7,200) .. controls (575.94,200) and (581,205.06) .. (581,211.3) -- (581,245.2) .. controls (581,251.44) and (575.94,256.5) .. (569.7,256.5) -- (491.3,256.5) .. controls (485.06,256.5) and (480,251.44) .. (480,245.2) -- cycle ;
    %Rounded Rect [id:dp2545445086969502]
    \draw   (118,212.3) .. controls (118,206.06) and (123.06,201) .. (129.3,201) -- (207.7,201) .. controls (213.94,201) and (219,206.06) .. (219,212.3) -- (219,246.2) .. controls (219,252.44) and (213.94,257.5) .. (207.7,257.5) -- (129.3,257.5) .. controls (123.06,257.5) and (118,252.44) .. (118,246.2) -- cycle ;
    %Right Arrow [id:dp4094129662689978]
    \draw   (229,225.86) -- (446.13,225.86) -- (446.13,218.5) -- (472,229.75) -- (446.13,241) -- (446.13,233.64) -- (229,233.64) -- cycle ;
    %Right Arrow [id:dp613459402399926]
    \draw   (294.1,69.22) -- (195.1,180.57) -- (200.6,185.46) -- (180.4,191.25) -- (183.79,170.51) -- (189.29,175.4) -- (288.29,64.05) -- cycle ;
    %Right Arrow [id:dp26885093244414104]
    \draw   (525.19,188.48) -- (419.52,83.43) -- (414.33,88.66) -- (409.67,68.17) -- (430.19,72.7) -- (425,77.92) -- (530.67,182.96) -- cycle ;

    % Text Node
    \draw (326,24) node [anchor=north west][inner sep=0.75pt]  [font=\large] [align=left] {Model};
    % Text Node
    \draw (149,219) node [anchor=north west][inner sep=0.75pt]  [font=\large] [align=left] {View};
    % Text Node
    \draw (491,217) node [anchor=north west][inner sep=0.75pt]  [font=\large] [align=left] {Controller};
    % Text Node
    \draw (154,102) node [anchor=north west][inner sep=0.75pt]   [align=left] {aktualisiert};
    % Text Node
    \draw (494,104) node [anchor=north west][inner sep=0.75pt]   [align=left] {manipuliert};
    % Text Node
    \draw (278,246) node [anchor=north west][inner sep=0.75pt]   [align=left] {sendet user eingabe};


\end{tikzpicture}
        \caption{Interaktionen im Model-View-Controller}
        \label{fig:mvc}
    \end{figure}

    Das \textbf{Model} repräsentiert die Struktur der Daten und hält Informationen über den aktuellen Zustand der Anwendung.
    Außerdem ist es verantwortlich für die Kommunikation mit externen Datenquellen wie beispielsweise Datenbanken.
    ~\cite{mvc}
    Es liefert die Daten an die View weiter, ohne diese zu verändern.
    Die Daten im Model sind somit unabhängig von ihrer Präsentation.
    Deshalb ist es möglich mehrere Views an dasselbe Model, und somit dieselben Daten zu binden.
    Hierbei sind jedoch die Präsentationen der Daten auf den einzelnen Views ebenfalls unabhängig voneinander.
    Sollten sich die Daten im Model ändern werden die Views über diese Änderung informiert.~\cite{https://doi.org/10.48550/arxiv.1408.5786}

    Die \textbf{View} stellt die optische Oberfläche der Anwendung dar.
    Sie enthält alle grafischen Elemente wie Textfelder, Bilder oder anderer Elemente die zur Darstellung der Daten aus dem Model benötigt werden.
    Zudem nimmt sie die Eingaben des Users entgegen und leitet diese an den Controller weiter.~\cite{https://doi.org/10.48550/arxiv.1408.5786}


    Der \textbf{Controller} ist für die Steuerung der Anwendung zuständig.
    Seine Aufgabe ist es je nach Eingabe des Benutzers das Model manipulieren.
    Basierend auf der Eingabe des Benutzers ist er ebenfalls dafür zuständig die aktuelle View zu wechseln.~\cite{https://doi.org/10.48550/arxiv.1408.5786}


    Im Model-View-Controller Pattern sind Programmlogik, Datenstruktur und die Präsentation der Daten derart getrennt, dass es möglich ist diese Unabhängig voneinander zu verändern, ohne damit Einfluss auf einen der anderen Bereiche zu nehmen.
    Somit ist es möglich beispielsweise die Oberfläche eines Programmes zu überarbeiten, ohne sich dabei Gedanken über die darunter liegende Ausführungslogik zu machen. ~\cite{mvc}

    \subsection{Model-View-Presenter}\label{subsec:model-view-presenter}
    Die Model-View-Presenter Architektur ist eine weiterentwicklung der Model-View-Controller Architektur und dieser somit sehr ähnlich.
    Sie ist ebenfalls in 3 Bereiche geteilt, die View und das Model übernehmen die gleichen Funktionieren wie auch schon in der Model-View-Controller Architektur.
    Entscheidender Unterschied hierbei ist jedoch, dass View und Controller nicht direkt miteinander verbunden sind.
    Wie in Abbildung \ref{fig:mvp} zu sehen werden Model und View über den Presenter verbunden.
    Dieser ersetzt den Controller und fungiert als Schnittstelle beider Komponenten.
    Die über die View eintreffenden Eingaben der Users werden vom Presenter verarbeitet und damit das jeweilige Model manipuliert.
    Das daraus resultierende aktualisierte Model wird dann vom Presenter zurück an die View geschickt um diese zu aktualisieren.~\cite{mvp1}
    \begin{figure}[h]
        \centering
        \tikzset{every picture/.style={line width=0.75pt}} %set default line width to 0.75pt

\begin{tikzpicture}[x=0.75pt,y=0.75pt,yscale=-1,xscale=1]
%uncomment if require: \path (0,300); %set diagram left start at 0, and has height of 300

%Rounded Rect [id:dp23143765937354965]
    \draw   (458.1,33.32) .. controls (458.1,27.08) and (463.16,22.02) .. (469.4,22.02) -- (547.8,22.02) .. controls (554.04,22.02) and (559.1,27.08) .. (559.1,33.32) -- (559.1,67.22) .. controls (559.1,73.46) and (554.04,78.52) .. (547.8,78.52) -- (469.4,78.52) .. controls (463.16,78.52) and (458.1,73.46) .. (458.1,67.22) -- cycle ;
%Rounded Rect [id:dp11583586667477586]
    \draw   (282,215.3) .. controls (282,209.06) and (287.06,204) .. (293.3,204) -- (371.7,204) .. controls (377.94,204) and (383,209.06) .. (383,215.3) -- (383,249.2) .. controls (383,255.44) and (377.94,260.5) .. (371.7,260.5) -- (293.3,260.5) .. controls (287.06,260.5) and (282,255.44) .. (282,249.2) -- cycle ;
%Rounded Rect [id:dp2545445086969502]
    \draw   (101,32.3) .. controls (101,26.06) and (106.06,21) .. (112.3,21) -- (190.7,21) .. controls (196.94,21) and (202,26.06) .. (202,32.3) -- (202,66.2) .. controls (202,72.44) and (196.94,77.5) .. (190.7,77.5) -- (112.3,77.5) .. controls (106.06,77.5) and (101,72.44) .. (101,66.2) -- cycle ;
%Right Arrow [id:dp26885093244414104]
    \draw   (271.34,212.89) -- (149.6,98.41) -- (144.82,103.49) -- (137.61,82.09) -- (159.41,87.98) -- (154.63,93.05) -- (276.38,207.54) -- cycle ;
%Right Arrow [id:dp6878896108284522]
    \draw   (173.65,83.31) -- (278.62,181.07) -- (283.64,175.68) -- (288.48,195.56) -- (268.31,192.15) -- (273.33,186.76) -- (168.35,89) -- cycle ;
%Right Arrow [id:dp9126614286809187]
    \draw   (391.07,205.95) -- (497.3,94.48) -- (492.26,89.67) -- (512.62,83.73) -- (507.67,104.35) -- (502.63,99.55) -- (396.4,211.02) -- cycle ;
%Right Arrow [id:dp012985148990955597]
    \draw   (483.24,90.08) -- (389.79,187.05) -- (395.09,192.16) -- (375.86,195.9) -- (378.89,176.55) -- (384.2,181.66) -- (477.64,84.69) -- cycle ;

% Text Node
    \draw (483,39) node [anchor=north west][inner sep=0.75pt]  [font=\large] [align=left] {Model};
% Text Node
    \draw (132,39) node [anchor=north west][inner sep=0.75pt]  [font=\large] [align=left] {View};
% Text Node
    \draw (293,221) node [anchor=north west][inner sep=0.75pt]  [font=\large] [align=left] {Presenter};
% Text Node
    \draw (134,166) node [anchor=north west][inner sep=0.75pt]   [align=left] {aktualisiert};
% Text Node
    \draw (460,150) node [anchor=north west][inner sep=0.75pt]   [align=left] {manipuliert};
% Text Node
    \draw (248,129) node [anchor=north west][inner sep=0.75pt]   [align=left] {User Eingabe};
% Text Node
    \draw (305,87) node [anchor=north west][inner sep=0.75pt]   [align=left] {Aktualisiertes Model};


\end{tikzpicture}

        \caption{Interaktionen im Model-View-Presenter}
        \label{fig:mvp}
    \end{figure}

    Durch diesen Aufbau haben View und Model klar definierte Schnittstellen, welche vom Presenter miteinander verbunden werden.
    Im Vergleich zur Model-View-Controller Architektur erhält man somit eine noch deutlichere Trennung der einzelnen Komponenten.~\cite{mvp2}

    \subsection{Model-View-ViewModel}\label{subsec:model-view-viewmodel}
    Die Model-View-ViewModel Architektur ist ebenfalls eine weiterentwicklung der Model-View-Controller Architektur.


    \begin{figure}[h]
        \centering
        

\tikzset{every picture/.style={line width=0.75pt}} %set default line width to 0.75pt

\begin{tikzpicture}[x=0.75pt,y=0.75pt,yscale=-1,xscale=1]
%uncomment if require: \path (0,354); %set diagram left start at 0, and has height of 354

%Rounded Rect [id:dp2545445086969502]
    \draw   (58,153.8) .. controls (58,145.07) and (65.07,138) .. (73.8,138) -- (143.2,138) .. controls (151.93,138) and (159,145.07) .. (159,153.8) -- (159,201.2) .. controls (159,209.93) and (151.93,217) .. (143.2,217) -- (73.8,217) .. controls (65.07,217) and (58,209.93) .. (58,201.2) -- cycle ;
%Rounded Rect [id:dp5590074384460382]
    \draw   (261.33,152.8) .. controls (261.33,144.07) and (268.41,137) .. (277.13,137) -- (367.2,137) .. controls (375.93,137) and (383,144.07) .. (383,152.8) -- (383,200.2) .. controls (383,208.93) and (375.93,216) .. (367.2,216) -- (277.13,216) .. controls (268.41,216) and (261.33,208.93) .. (261.33,200.2) -- cycle ;
%Rounded Rect [id:dp3462199580111829]
    \draw   (481,153.8) .. controls (481,145.07) and (488.07,138) .. (496.8,138) -- (566.2,138) .. controls (574.93,138) and (582,145.07) .. (582,153.8) -- (582,201.2) .. controls (582,209.93) and (574.93,217) .. (566.2,217) -- (496.8,217) .. controls (488.07,217) and (481,209.93) .. (481,201.2) -- cycle ;
%Straight Lines [id:da9849453642036357]
    \draw    (159.33,176) -- (259.33,176) ;
    \draw [shift={(261.33,176)}, rotate = 180] [color={rgb, 255:red, 0; green, 0; blue, 0 }  ][line width=0.75]    (10.93,-3.29) .. controls (6.95,-1.4) and (3.31,-0.3) .. (0,0) .. controls (3.31,0.3) and (6.95,1.4) .. (10.93,3.29)   ;
%Straight Lines [id:da21648687729164617]
    \draw    (261.33,176) -- (161.33,176) ;
    \draw [shift={(159.33,176)}, rotate = 360] [color={rgb, 255:red, 0; green, 0; blue, 0 }  ][line width=0.75]    (10.93,-3.29) .. controls (6.95,-1.4) and (3.31,-0.3) .. (0,0) .. controls (3.31,0.3) and (6.95,1.4) .. (10.93,3.29)   ;
%Straight Lines [id:da6547359698670925]
    \draw    (383.33,169) -- (477.33,169) ;
    \draw [shift={(479.33,169)}, rotate = 180] [color={rgb, 255:red, 0; green, 0; blue, 0 }  ][line width=0.75]    (10.93,-3.29) .. controls (6.95,-1.4) and (3.31,-0.3) .. (0,0) .. controls (3.31,0.3) and (6.95,1.4) .. (10.93,3.29)   ;
%Straight Lines [id:da6872128973181653]
    \draw    (481.33,187) -- (385.33,187) ;
    \draw [shift={(383.33,187)}, rotate = 360] [color={rgb, 255:red, 0; green, 0; blue, 0 }  ][line width=0.75]    (10.93,-3.29) .. controls (6.95,-1.4) and (3.31,-0.3) .. (0,0) .. controls (3.31,0.3) and (6.95,1.4) .. (10.93,3.29)   ;

% Text Node
    \draw (89,166.79) node [anchor=north west][inner sep=0.75pt]  [font=\large] [align=left] {View};
% Text Node
    \draw (272,166.79) node [anchor=north west][inner sep=0.75pt]  [font=\large] [align=left] {ViewModel};
% Text Node
    \draw (505,166.79) node [anchor=north west][inner sep=0.75pt]  [font=\large] [align=left] {Model};
% Text Node
    \draw (162,150) node [anchor=north west][inner sep=0.75pt]   [align=left] {data binding};
% Text Node
    \draw (388,145) node [anchor=north west][inner sep=0.75pt]   [align=left] {manipulates};
% Text Node
    \draw (408,193) node [anchor=north west][inner sep=0.75pt]   [align=left] {notifies};


\end{tikzpicture}

        \caption{Interaktionen im Model-View-Presenter}
        \label{fig:mvvm}
    \end{figure}


    bdfsal;fas dfas


    \chapter{Verwendete Technologien}\label{ch:verwendete-technologien}
    In diesem Kapitel werden die für die Implementierung der Plattform verwendeten Technologien vorgestellt.



    \section{XCode}\label{sec:xcode}


    \section{Simulator}\label{sec:simulator}




    \section{Swift}\label{sec:swift}

    \newpage
    \section{Cocoapods}\label{sec:cocoapods}
    CocoaPods\footnote{https://github.com/CocoaPods/CocoaPods} ist ein Dependency Manager für Objective-C und Swift Projekte.
    Es ermöglicht dem Entwickler Quellcode von verschiedenen Orten in ein Projekt einzubinden.
    Hierzu können über 92.000 Libraries verwendet werden.
    Besonders im Bereich von Mac OSX und iOS Development findet das Tool häufig Verwendung.
    CocoaPods unterstützt sowohl private als auch öffentliche Repositories über git, svn, brz, http und hg.
    Hierbei ist zu beachten, dass das nicht jedes Repository automatisch über Cocoapods eingebunden werden kann.
    Es muss zuerst ein Pod aus der jeweiligen Library erstellt werden.
    Das seit 2014 entwickelte Projekt ist Open Source und wurde seit dem von über 300 Entwicklern stetig verbessert.
    Es ist ein Kommandozeilen-Tool welches in Ruby implementiert wird.
    Somit kann es über den RubyGems Paketmanager heruntergeladen werden.
    Hierzu kann nach erfolgreicher Installation von Ruby folgendes Kommando in der Konsole ausgeführt werden:
    \begin{lstlisting}[language=bash,label={lst:cocoapods}]
        $ sudo gem install cocoapods
    \end{lstlisting}
    Mittels CocoaPods lassen sich dann alle benötigten Libraries im sogenannten Podfile definieren und importieren.
    Das Podfile ist eine einfache Textdatei in welcher die Zielplattform und eine Liste aller benötigter Libraries und ihrer jeweiligen Versionen angegeben wird.
    In Listing\ref{lst:podfile} ist die Struktur eins solchen Podfiles zu sehen.
    ~\cite{cocoapods1, cocoapods2, cocoapods3}
    \begin{lstlisting}[language={},firstnumber=1,label={lst:podfile},caption={Beispielstuktur eines Podfiles},captionpos=t]
platform :ios, '8.0'
use_frameworks!

target 'MyApp' do
  pod 'AFNetworking', '~> 2.6'
  pod 'ORStackView', '~> 3.0'
  pod 'SwiftyJSON', '~> 2.3'
end
    \end{lstlisting}


    \section{Matrix iOS SDK}\label{sec:matrix-sdk}
    Die Matrix iOS SDK \footnote{https://github.com/matrix-org/matrix-ios-sdk} ist eine Open Source Library welche zur für die Entwicklung Matrix basierter Anwendungen für iOS Geräte gedacht ist.
    Sie fungiert als Schnittstelle zwischen Client Application und Matrix Server.
    Hierzu werden die in der Client-Server-API definierten Operationen bereitgestellt.
    Um die Matrix SDK in ein Projekt einzubinden muss hierfür ein Eintrag im in Abschnitt \ref{sec:cocoapods} erläuterten Podfile angelegt und installiert werden.
    ~\cite{matrixiossdk}
    \begin{lstlisting}[language={},label={lst:matrtix-sdk}]
        pod 'SwiftMatrixSDK'
    \end{lstlisting}
    Anschließend kann die Library im Swift Code importiert werden.
    \begin{lstlisting}[language=swift,label={lst:matrtix-sdk-swift}]
        import MatrixSDK
    \end{lstlisting}

    \section{Docker}\label{sec:docker}
    Docker ist eine Containervirtualisierungssoftware welche es ermöglicht Anwendung vom Rest des Systems zu isolieren.
    Auch wenn sie nach außen wie eine virtuelle Maschine wirkt, bietet sie entscheidende Vorteile.
    Herkömmliche virtuelle Maschinen bilden eine vollständige Kopie eines Betriebssystems welches mittels einer Virtualisierungssoftware wie beispielsweise Oracle VirtualBox oder KVM auf einem Hypervisor betrieben werden muss.
    Betrachtet man den vollständigen Stack einmal von unten hat man zuallererst die Hardware des Hypervisors, auf diesem läuft ein Betriebssystem, worauf wiederum die Virtualisierungssoftware läuft, mit welcher eine isolierte Kopie eines anderen Betriebssystems erzeugt wird in welcher die eigentliche Anwendung dann läuft.
    Dies sorgt für erhebliche Einbußen in der Performance.
    Docker Container sind in der Lage in wenigen Sekunden zu starten.
    Sie laufen direkt auf dem Betriebssystem des Hypervisors sind aber dennoch isoliert vom Rest des Systems.
    Dies geschieht mithilfe zweier Linux Kernel Technologien, namespaces und cgroups.
    ~\cite{docker}

    \section{Synapse}\label{sec:synapse}

    \chapter{Konzept}\label{ch:konzept}
    In diesem Kapitel werden 2 verwandte Projekte analysiert und unter Betrachtung der Zielgruppe einige Anforderungen definiert die die App zu erfüllen hat.

    \section{Analyse ähnlicher Projekte}\label{sec:analyse-ahnlicher-projekte}

    \subsection{TI-Messenger}


    \subsection{Nio}
    Nio\footnote{https://github.com/niochat/nio} Client



    \newpage
    \section{Anforderungen}\label{sec:anforderungsliste}
    Nach der Analyse ähnlicher Projekte wurden eine Liste funktionaler Anforderungen erstellt.
    Diese wurden nach der MoSCoW-Methode priorisiert.\\

    \textbf{Must}:
    \begin{enumerate}[label={\roman*.}, leftmargin=2.5cm]
        \item Der User muss über die App einen Account auf der Plattform anlegen können.
        \item Der User muss sich mit seinem Account in der App einloggen können.
        \item Der User muss sein Passwort zu ändern können.
        \item Dem User muss alle von beigetretenen Räume sehen können.
        \item Dem User muss neue Räume erstellen können.
        \item Der User muss zu neuen Räumen eingeladen werden können.
        \item Der User muss alle Nachrichten die in einem Raum gesendet wurden einzusehen können.
        \item Der User muss Nachrichten in einem Raum zu senden können.
        \item Nachrichten zwischen Usern müssen End-to-End Verschlüsselt sein.
        \item Der User muss einen Raum verlassen können.
        \item Der User muss seinen Account deaktivieren können.
    \end{enumerate}


    \textbf{Should}:
    \begin{enumerate}[label={\roman*.}, leftmargin=2.5cm]
        \item Beim erstellen des Accounts soll eine zusätzliche Authentifizierungsmethode verwendet werden um Wahloses erstellen zon Accounts zu verhindern.
        \item Der User soll sich nur einmal einloggen müssen.
        \item Der User soll sich ausloggen können.
        \item Der Benutzer soll die Möglichkeit haben sein Profilbild und seinen Anzeigenamen anzupassen.
        \item Der vollständige Chat-Verlauf soll nur bei Bedarf geladen werden.
        \item Die App soll neben Textnachrichten auch andere Nachrichtentypen wie Fotos oder Videos unterstützen.
        \item Die App soll den User über den Erhalt einer neuen Nachricht informieren.
        \item Die Übersicht der beigetretenen Räume soll nach letzter Aktivität sortiert werden.
        \item Die App soll Räume und Chat-Verläufe local speichern und offline wiedergeben können.
    \end{enumerate}


    \textbf{Could}:
    \begin{enumerate}[label={\roman*.}, leftmargin=2.5cm]
        \item Die Liste der beigetretenen Räume kann gefiltert werden.
        \item Dem Erstellen eines Raumes kann dem User eine Liste von Usern vorgeschlagen werden, welche dem gesuchten Namen entsprechen.
        \item Dem User kann ein Typing-Indikator gezeigt werden.
        \item Die App kann auch mit anderen Homeservern verbunden werden.
        \item Der User kann den Inhalt einer Textnachricht in die Zwischenablage kopieren.
        \item Der User kann eine Nachricht weiterleiten.
        \item Der User kann Dateien welche in einem Raum verschickt wurden herunterladen.
    \end{enumerate}


    \section{Auswahl der Systemarchitektur}\label{sec:auswahl-der-systemarchitektur}

    \newpage
    \chapter{Implementierung}\label{ch:implementierung}
    Dieses Kapitel beschreibt die 3



    \section{User Interface}\label{sec:user-interface}
    Der erste Schritt in der Entwicklung der App ist das Design des User Interfaces.
    Hierbei sollte auf
    Dieses besteht aus mehreren einzelner Views welche welche sich zu verschiedenene Gruppen zusammenführen lassen.

    \section{Homeserver}\label{sec:homeserver}

    Nachdem die App das erste mal gestartet wird
    \newpage
    \section{Account Erzeugungs Flow}\label{sec:account-erzeugungs-flow}
    Um die App nutzen zu können muss zualler erst einmal ein Account erstellt werden können.
    Hierzu können aufseiten des Homeservers verschiedene Authentifizierungsstufen definiert werden, welche vom Client durchlaufen werden müssen, um einen Account zu erstellen.
    Diese können beispielseise
    In Abbildung~\ref{fig:accountCreationDiagram} ist ein dreistufiger Accounterzugungsvorgang mittels eines Registriersungstokes beschrieben.
    Dieser registrierungstoken kann entweder in der Datenbank direkt aangelegt werden oder mittels eines Administartor users auf der admin API erstellt werden.
    Dieser token muss dem User anschließend zur Verfügung gestellt werden beispielsweise via email.
    Der eigentliche Autehntifizierungsprozess beginnt dann wie folgt.
    Im ersten Schritt sendet der Client eine leere Anfrage zum Server in welcher er um das starten einer registrierungs session bittet.
    Der Server antortet daraufhin mit einer SessionID und den vom Homeserver unterstützten Authentifizierungs Flows.
    In diesem Beispiel ist der einzige unterstützte Flow die verifizierung über einen Registriersungstoken.
    Im nächsten Schritt sendet der Client dann den zuvor erstellten registrierungstoken und die vom server bereitgestellte sessionid in Kombination mit der aktuellen Stufe des Authentifizierungs prozessen.
    Im dritten und letzten Schritt wird nun der dummyflow aufgeführt.
    Dieser Flow kann nicht fehlschlagen und dient der Letzendlichen erstellung des Account.
    Hierbei werden der gewünsche username und passwort übergeben.
    Ist dies gesehen wurde der Account erfolgreich erstellt.

    \begin{figure}[h]
        \centering
        \begin{sequencediagram}
            \newthread{A}{Client}{}
            \newinst[7]{B}{Server}{}
            \begin{sdblock}{Stage 1}{Receive Session ID}
                \begin{call}{A}{getRegistrationSession()}{B}{\shortstack{
                    return Session ID\\
                    return supported flows}}
                    \postlevel
                \end{call}
            \end{sdblock}
            \begin{sdblock}{Stage 2}{m.login.registration\_token}
                \begin{call}{A}{register()}{B}{}
                \end{call}
            \end{sdblock}
            \begin{sdblock}{Stage 3}{m.login.dummy}
                \begin{call}{A}{register()}{B}{}
                \end{call}
            \end{sdblock}
        \end{sequencediagram}
        \caption{Account erzeugungs Flow}
        \label{fig:accountCreationDiagram}
    \end{figure}

    \section{Raum erstellung}\label{sec:raum-erstellung}
    User directory

    \newpage

    \subsection{Text}\label{subsec:text}
    Der wohl wichtigste Nachrichtentyp in
    \subsection{Foto}\label{subsec:foto}
    text
    \subsection{Video}\label{subsec:video}
    text
    \subsection{Datei}\label{subsec:datei}
    text

    \section{Account Deactivation Flow}\label{sec:account-deactivation-flow}
    text

    \begin{figure}[h]
        \centering
        \begin{sequencediagram}
            \newthread{A}{Client}{}
            \newinst[7]{B}{Server}{}
            \begin{sdblock}{Stage 1}{Receive Session ID}
                \begin{call}{A}{getLoginSession()}{B}{\shortstack{
                    return Session ID\\
                    return supported flows}}
                    \postlevel
                \end{call}
            \end{sdblock}
            \begin{sdblock}{Stage 2}{m.login.password}
                \begin{call}{A}{deactivateAccount()}{B}{}
                \end{call}
            \end{sdblock}
        \end{sequencediagram}
        \caption{Account löschen Flow}
    \end{figure}

    \section{Nachrichten}\label{sec:nachrichten}
    some text

    \begin{lstlisting}[language=json,firstnumber=1]
{"menu": {
  "id": "file",
  "value": "File",
  "popup": {
    "menuitem": [
      {"value": "New", "onclick": "CreateNewDoc()"},
      {"value": "Open", "onclick": "OpenDoc()"},
      {"value": "Close", "onclick": "CloseDoc()"}
    ]
  }
}}
    \end{lstlisting}

    \newpage
    \section{Benachrichtigungen}\label{sec:benachrichtigungen}

    \newpage
    \chapter{Evaluation}
    \chapter{Fazit}\label{ch:fazit}
    Zwar wurde während der Implementierung darauf geachtet, dass jegliche Daten zwischen Arzt und Patient verschlüsselt und vor unbefugtem Zugriff geschützt sind, jedoch müsste die genaue Rechtslage zum Einsatz der Plattform genauer untersucht werden.

    \chapter{Ausblick}\label{ch:ausblick}

%\include{source/content/Test}

%Das Fazit
%\include{source/content/Fazit}
%Einbinden des Abbildungsverzeichnisses

    \backmatter
%Liste der Tabellen
    \listoftables
%Einbinden des Tabellenverzeichnisses
    \listoffigures
%Einbinden des Sourcecodeverzeichnisses
%\lstlistoflistings

% Quellenverzeichnis
    \bibliographystyle{abbrvnat}
    \bibliography{Bachelorarbeit}

% Anhang
    \appendix
\end{document}



%
% EOF
%